\title{Deriving OMI slit functions using its solar irradiance and testing the implications to retrievals}

\author{Kang Sun}
\documentclass[hidelinks,12pt]{article}
\usepackage{amsmath}
\usepackage[letterpaper,left=.75in,right=.75in,top=.75in,bottom=.75in]{geometry}
\usepackage{setspace}
\usepackage{lineno}
\newcommand{\beginsupplement}{%
        \setcounter{table}{0}
        \renewcommand{\thetable}{S\arabic{table}}%
        \setcounter{figure}{0}
        \renewcommand{\thefigure}{S\arabic{figure}}%
     }
\usepackage[section]{placeins}

\usepackage{threeparttable,booktabs}
\usepackage[square]{natbib}
\usepackage{longtable}
\usepackage{multirow}
\usepackage{graphicx}
\usepackage[space]{grffile}
\usepackage{color}
\usepackage{hyperref}
\hypersetup{%
    pdfborder = {0 0 0}
}
%\usepackage{etoolbox}
\usepackage{gensymb}
%\usepackage{subcaption}
%\appto\TPTnoteSettings{\footnotesize}


\begin{document}
\maketitle
\abstract
This report documents the effort to update OMI the slit function parameterization and investigates the impacts on ozone profile retrievals (so far). Super Gaussian slit function has been added as an option to model the slit functions.

\section{Technicality: modification to the codes}

A new .f90 file was added (\textbf{super\_gauss.f90}, which includes subroutines that define the Super Gaussian slit function and how to convolve it. Its structure is very similar to \textbf{gauss.f90}, \textbf{voigt.f90}, and \textbf{triangle.f90}.

Implementing Super Gaussian requires an additional parameter, i.e., the parameter $k$ in 
\begin{equation*}
S(\Delta\lambda) = \frac{k}{2w\Gamma(1/k)}\exp \left(-\left|\frac{\Delta\lambda}{w}\right|^k \right).
\end{equation*}
Therefore, the SOMIPROF.inp file needs to be modified to add the initial/boundaries of this parameter, called ``spk'' (stands for ``super k''). 


To accomodate this new slit function option and one additional fitting parameter, many .f90 source codes need to be modified, including:

\textbf{OMSAO\_indices\_module, cal\_fit\_one, omi\_cross\_calibrate, rad\_fit\_vary, rad\_wavcal\_vary, radiance\_wavcal, solar\_fit, solar\_fit\_vary, solar\_wavcal\_vary, spectra, undersample, dataspline, and omi\_pge\_fitting\_aux} in src/omi-pge, and

\textbf{get\_raman\_chliu, spectra\_reflectance, and lidort\_prof\_utilities} in src/o3prof.

I want to especially emphasize that the dimension of variable ``tmpslit'' in \textbf{solar\_wavcal\_vary} and \textbf{rad\_wavcal\_vary} needs to be increased from 6 to 7 due to the addition of fitting variable ``spk''.

Also the newly added \textbf{super\_gauss.f90} needs to be included in \textbf{make/make.patterns} and \textbf{make/make.sources}.

When migrating these changed source codes to Hydra, the only problem that poped out was \textbf{omi\_pge\_fiting\_aux}. Apparently there are some differences between the local and Hydra verisons of this code that I don't understand. This code was just mannually changed based on the native Hydra version.

By the way, the updated slit function options are:

0: Symmetric Gauss 1: Asymmetric Gauss  2. Voigt  3: Tirangle (symmetric) 4: OMI-preflight slit 9: Super Gaussian

\section{Impacts of slit functions on ozone profile retrievals}

\subsection{Data}
The test data I used is from OMI orbit 05311 on 14 July 2005. Lines 680--710 and all cross-track positions were included.

The soundings were from tropical Pacific Ocean, with some clouds towards the edges.% The effective cloud pressures look mysterious though.

\begin{figure}[hbtp]
  \begin{center}
    \includegraphics[width=.6\linewidth]{/data/tempo1/Shared/kangsun/OMI/figures/ozprof_slit/clouds_no_coadd_run.pdf}
  \caption{Location of test dataset and cloud coverage.}
    \label{fig1}
  \end{center}
\end{figure}

\subsection{Results}
Besides the functional form of slit functions, the order of coadding and wavelength calibration turned out to be a critical factor. This effect is the most significant when using preflight slit functions, which was defined at 60 cross-track positions at UV2. If wavcal is done before coadding, the native preflight data can be used. Otherwise, the average preflight parameters of adjacent cross-track positions are used to calculate the preflight slit function defined at the 30 cross-track positions after coadding.

Figures~\ref{fig2} and \ref{fig3} show the UV2 fitting residuals when coadding first, then wavlcal, and vice versa.

Figures~\ref{fig4} and \ref{fig5} show the UV1 fitting residuals when coadding first, then wavlcal, and vice versa.

Figures~\ref{fig6} and \ref{fig7} show the differences between various slit functions and the baseline (standard Gaussian) when coadding first, then wavlcal, and vice versa.

Figures~\ref{fig8} and \ref{fig9} show the same data as figures~\ref{fig6} and \ref{fig7}, but with more clear visualization of vertical profiles.

\begin{figure}[hbtp]
  \begin{center}
    \includegraphics[width=1\linewidth]{/data/tempo1/Shared/kangsun/OMI/figures/ozprof_slit/rms2_coadd_run.pdf}
  \caption{The UV2 fitting residuals when coadding first, then wavlcal.}
    \label{fig2}
  \end{center}
\end{figure}


\begin{figure}[hbtp]
  \begin{center}
    \includegraphics[width=1\linewidth]{/data/tempo1/Shared/kangsun/OMI/figures/ozprof_slit/rms2_no_coadd_run.pdf}
  \caption{The UV2 fitting residuals when wavecal first, then coadding.}
    \label{fig3}
  \end{center}
\end{figure}


\begin{figure}[hbtp]
  \begin{center}
    \includegraphics[width=1\linewidth]{/data/tempo1/Shared/kangsun/OMI/figures/ozprof_slit/rms1_coadd_run.pdf}
  \caption{The UV1 fitting residuals when coadding first, then wavlcal.}
    \label{fig4}
  \end{center}
\end{figure}


\begin{figure}[hbtp]
  \begin{center}
    \includegraphics[width=1\linewidth]{/data/tempo1/Shared/kangsun/OMI/figures/ozprof_slit/rms1_no_coadd_run.pdf}
  \caption{The UV1 fitting residuals when wavecal first, then coadding.}
    \label{fig5}
  \end{center}
\end{figure}


\begin{figure}[hbtp]
  \begin{center}
    \includegraphics[width=1\linewidth]{/data/tempo1/Shared/kangsun/OMI/figures/ozprof_slit/prof_diff_lines_coadd_run.pdf}
  \caption{The differences between various slit functions and the baseline (standard Gaussian) when coadding first, then wavlcal.}
    \label{fig6}
  \end{center}
\end{figure}


\begin{figure}[hbtp]
  \begin{center}
    \includegraphics[width=1\linewidth]{/data/tempo1/Shared/kangsun/OMI/figures/ozprof_slit/prof_diff_lines_no_coadd_run.pdf}
  \caption{The differences between various slit functions and the baseline (standard Gaussian) when wavecal first, then coadding.}
    \label{fig7}
  \end{center}
\end{figure}


\begin{figure}[hbtp]
  \begin{center}
    \includegraphics[width=1\linewidth]{/data/tempo1/Shared/kangsun/OMI/figures/ozprof_slit/prof_diff_pcolor_coadd_run.pdf}
  \caption{The differences between various slit functions and the baseline (standard Gaussian) when coadding first, then wavlcal.}
    \label{fig8}
  \end{center}
\end{figure}


\begin{figure}[hbtp]
  \begin{center}
    \includegraphics[width=1\linewidth]{/data/tempo1/Shared/kangsun/OMI/figures/ozprof_slit/prof_diff_pcolor_no_coadd_run.pdf}
  \caption{The differences between various slit functions and the baseline (standard Gaussian) when wavecal first, then coadding.}
    \label{fig9}
  \end{center}
\end{figure}
\FloatBarrier
\section{Conclusions}
\begin{enumerate}
\item Codes are ready for more tests.
\item Coadding before wavelength calibration is not fair for preflight slit functions, although cross-track-dependent bias of using preflight is significant either way.
\item Something is not quite right when using asymmetric Gaussian in the wavcal then coadding case.
\item Coadding complicates things. Next step is to only use UV2.
\end{enumerate}

\end{document}


\section{Conclusions}
\begin{enumerate}
\item Super Gaussian is nice.
\item Linearizing the fitting using pseudo absorber is reasonably accurate if you know what you are doing.
\item Be careful with wavelength calibration using OMI preflight slit functions.
\end{enumerate}
\end{document}

